% planisphere.tex
%
% The LaTeX code in this file brings together into a single document the
% various components of the model planisphere.
%
% Copyright (C) 2014-2019 Dominic Ford <dcf21-www@dcford.org.uk>
%
% This code is free software; you can redistribute it and/or modify it under
% the terms of the GNU General Public License as published by the Free Software
% Foundation; either version 2 of the License, or (at your option) any later
% version.
%
% You should have received a copy of the GNU General Public License along with
% this file; if not, write to the Free Software Foundation, Inc., 51 Franklin
% Street, Fifth Floor, Boston, MA  02110-1301, USA

% ----------------------------------------------------------------------------

\documentclass[a4paper,onecolumn,10pt]{article}
\usepackage[dvips]{graphicx}
\usepackage{fancyhdr,url}
\usepackage{parskip}
\usepackage[pdftitle={Φτιάξτε το δικό σας επιπεδόσφαιρο}, pdfauthor={Dominic Ford}, pdfsubject={Build your own planisphere}, pdfkeywords={Φτιάξτε το δικό σας επιπεδόσφαιρο}, colorlinks=true, linkcolor=blue, citecolor=blue, filecolor=blue, urlcolor=blue]{hyperref}
\renewcommand{\familydefault}{\sfdefault}
\pagestyle{fancy}

\lhead{\it ΦΤΙΑΞΤΕ ΤΟ ΔΙΚΟΣ ΣΑΣ ΕΠΙΠΕΔΟΣΦΑΙΡΟ}
\chead{}
\rhead{\thepage}
\lfoot{}\rfoot{}
\cfoot{\bf\footnotesize\copyright\ 2014--2019 Dominic Ford. Distributed under the GNU General Public License, version 3. Document downloaded from \url{https://in-the-sky.org/planisphere/}}

\fancypagestyle{plain}{%
\fancyhf{} % clear all header and footer fields
\renewcommand{\headrulewidth}{0pt}
\renewcommand{\footrulewidth}{0pt}}

\title{Φτιάξτε το δικό σας επιπεδόσφαιρο}
\author{Dominic Ford}
\date{2014--2019}

\begin{document}
\maketitle
\setcounter{footnote}{1}

Το επιπεδόσφαιρο είναι μια απλή συσκευή που εμφανίζει ένα χάρτη με πια άστρα είναι
ορατά στο νυχτερινό ουρανό μια συγκεκριμένη στιγμή. Περιστρέφοντας έναν τροχό, εμφανίζει
το πως κινούνται τα άστρα στον ουρανό κατά τη διάρκεια της νύχτας, και πως διαφορετικοί 
αστερισμό είναι ορατοί σε διάφορες εποχές του χρόνου.

Παρουσιάζω εδώ ένα κιτ που μπορείτε να κατεβάσετε και να τυπώσετε για να φτιάξετε το δικό σας
επιπεδόσφαιρο από χαρτί ή χαρτόνι.

Η σχεδίαση του επιπεδόσφαιρου εξαρτάται από τη γεωγραφική θέση που θα χρησιμοποιηθεί, αφού 
διαφορετικά αστέρια είναι ορατά από διαφορετικές τοποθεσίες. Έχω δημιουργήσει εκδοχές 
για χρήση σε διαφορετικά γεωγραφικά πλάτη, τα οποία μπορείτε να κατεβάσετε από εδώ

\url{https://in-the-sky.org/planisphere/}

Το επιπεδόσφαιρο που βλέπετε σε αυτό το έγγραφο είναι σχεδιασμένο για χρήση στο γεωγραφικό πλάτος των 
\input{tmp/lat}.
 
\section*{Τι χρειάζεστε}

\begin{itemize}
\item Δύο κόλλες A4, καλύτερα να είναι από χαρτόνι.
\item Ψαλίδι.
\item Ένα διπλόκαρφο.
\item Προαιρετικά: ένα φύλλο διαφανούς πλαστικού, π.χ. διαφάνεια για χρήση με προβολέα.
\item Προαιρετικά: Λίγη κόλλα.
\end{itemize}

\section*{Οδηγίες Συναρμολόγησης}

{\bf Βήμα 1ο} -- Τα επιπεδόσφαιρα είναι διαφορετικά ανάλογα με το που ζείτε. 
Το επιπεδόσφαιρο αυτού του εγγράφου είναι σχεδιασμένο για χρήση οπουδήποτε στη Γη
που είναι λίγες μοίρες από το γεωγραφικό πλάτος \input{tmp/lat}. Αν ζείτε αλλού,
θα πρέπει να κατεβάσετε ένα κατάλληλο επιπεδόσφαιρο από τη σελίδα 

\url{https://in-the-sky.org/planisphere/}

{\bf Βήμα 2ο} -- Εκτυπώστε τις σελίδες αυτού του αρχείου PDF, που δείχνουν τον
τροχό των αστεριών και το σώμα του επιπεδόσφαιρου, σε δύο ξεχωριστές κόλλες χαρτί,
ή ακόμα καλύτερα σε παχύ χαρτόνι.

{\bf Βήμα 3ο} -- Κόψτε προσεκτικά τον τροχό των αστεριών και το σώμα του
επιπεδόσφαιρου. Κόψτε, επίσης, την γκρίζα περιοχή του σώματος του επιπεδόσφαιρου, και
αν μπορείτε, το πλέγμα των γραμμών που εκτυπώσατε στο διαφανές πλαστικό. 
Αν χρησιμοποιείτε χαρτόνι, ίσως να θέλετε να χαράξετε προσεκτικά το σώμα
του επιπεδόσφαιρου κατά μήκος της γραμμής με τις τελίτσες για να είναι πιο εύκολο το δίπλωμα
κατά μήκος αυτής γραμμή αργότερα.

{\bf Βήμα 4ο} -- Ο τροχός των αστεριών έχει ένα μικρό κύκλο στο κέντρο του, και το σώμα 
του επιπεδόσφαιρου έχει έναν αντίστοιχο μικρό κύκλο στο κάτω μέρος του. κάντε μια μικρή τρύπα
(περίπου 2mm) στον καθένα. Αν έχετε ένα τρυπάνι για χαρτί θα είναι ιδανικό,
αλλιώς χρησιμοποιείστε ένα διαβήτη και μεγαλώστε την τρύπα γυρνώντας τον σε κυκλική κίνηση.

{\bf Βήμα 5ο} -- Βάλτε ένα δίκαρφο από το μέσο του τροχού των αστεριών, 
με το κεφάλι του δίκαρφου από τη μεριά της εκτυπωμεης πλευράς του τροχού των αστεριών. 
Βάλτε μετά το σώμα του επιπεδόσφαιρου στο ίδιο δίκαρφο, με την εκτυπωμένη πλευρά
να κοιτά το πίσω μέρος του δίκαρφου. Διπλώστε το δίκαρφο για να ασφαλίσετε τα 
δύο φύλλα χαρτιού μαζί.

{\bf Βήμα 6ο (Προεραιτικό)} -- Αν εκτυπώσατε την τελική σελίδα αυτού του αρχείου PDF
σε διαφάνεια, πρέπει τώρα να την κολλήσετε πάνω από το παράθυρο προβολής
που κόψατε από το σώμα του επιπεδόσφαιρου.

{\bf Βήμα 7ο} -- Διπλώστε το σώμα του επιπεδόσφαιρου κατά μήκος της γραμμής με τις τελείες,
έτσι ώστε το εμπρός μέρος του τροχού των αστεριών να δείχνει μέσα από το παράθυρο που κόψατε 
από το σώμα.

{\bf Συγχαρητήρια, το επιπεδόσφαιρό σας είναι έτοιμο!}

\section*{Πως να χρησιμοποιήσετε το επιπεδόσφαιρο}

Γυρίστε τον τροχό των αστεριών μέχρι να βρείτε το σημείο στο άκρο του όπου είναι σημειωμένη
η σημερινή ημερομηνία, και ευθυγραμμίστε το σημείο αυτό με την τρέχουσα ώρα. Το παράθυρο προβολής 
εμφανίζει τώρα όλους τους αστερισμούς που είναι ορατοί στον ουρανό.

Βγείτε έξω και κοιτάξτε προς το βορρά. Κρατώντας το επιπεδόσφαιρο προς τον ουρανό, τα αστέρια που 
βρίσκονται στο κάτω μέρος του παραθύρου προβολής πρέπει να είναι αυτά που βλέπετε στον ουρανό
μπροστά σας.

Γυρίστε προς την ανατολή ή τη δύση, και περιστρέψτε το επιπεδόσφαιρο έτσι ώστε η λέξη 
"Ανατολή" ή "Δύση" να βρίσκεται στο κάτω μέρος του παραθύρου. Και πάλι τα αστέρια στο 
κάτω μέρος του παραθύρου προβολής πρέπει να ταιριάζουν με αυτά που βλέπετε στον ουρανό μπροστά σας.

Αν εκτυπώσατε τον ιστό των γραμμών γεωγραφικού πλάτους και μήκους στο διαφανές πλαστικό,
αυτές οι γραμμές σας επιτρέπουν να δείτε πως αντικείμενα ψηλά στον ορίζοντα 
θα εμφανιστούν στον ουρανό, και σε ποια κατεύθυνση. 
Οι κύκλοι είναι σχεδιασμένοι σε ύψος των 10, 20, 30, ..., 80 μοιρών πάνω από τον ορίζοντα. 
Μια απόσταση δέκα μοιρών ισούται χοντρικά με μια 
παλάμη αν έχουμε απλωμένο το χέρι μας. Οι κυρτές γραμμές είναι κάθετες 
που ενώνουν σημεία στον ορίζοντα μέχρι το σημείο αμέσως πάνω από το κεφάλι σας.
Είναι σχεδιασμένες προς τα βασικά σημεία του ορίζοντα Ν, ΝΝΑ, ΝΑ, ΑΝΑ, Α, κοκ.

\section*{Προσαρμοσμένα επιπεδόσφαιρα}

Το επιπεδόσφαιρο αυτό σχεδιάστηκε χρησιμοποιώντας μια συλλογή προγραμμάτων Python και τη
βιβλιοθήκη γραφικών pycairo. Αν θέλετε να προσαρμόσετε το επιπεδόσφαιρό σας, είστε
ευπρόσδεκτοι να κατεβάσετε τα προγράμματα από το λογαριασμό μου στο GitHub και να τα τροποποιήσετε,
αρκεί να αναφέρετε την πηγή:

\url{https://github.com/dcf21/planisphere}

\section*{Άδεια χρήσης}

Όπως ό,τιδήποτε άλλο στο {\tt In-The-Sky.org}, αυτά τα επιπεδόσφαιρα είναι 
\copyright\ του Dominic Ford. Παρ' όλα αυτά, τα πάντα στο {\tt In-The-Sky.org} 
παρέχονται προς όφελος των ερασιτεχνών αστρονόμων ανά την υφήλιο, και είστε ευπρόσδεκτοι
να τροποποιήσετε και/ή να αναδιανέμετε οποιοδήποτε υλικό αυτού του ιστοτόπου, υπό τους
ακόλουθους όρους: (1) Κάθε αντικείμενο που έχει σχετικό κείμενο πνευματικής {\bf
πρέπει} να περιλαμβάνει {\bf αναλλοίωτο} το κείμενο στην αναδιανεμημένη έκδοσή σας. (2)
{\bf Πρέπει} να αναφέρετε εμένα τον, Dominic Ford, ως τον αρχικό συγγραφέα και κάτοχο 
πνευματικών δικαιωμάτων. (3) {\bf Δεν μπορείτε} να αποκομίσετε οποιοδήποτε όφελος 
από την αναπαραγωγή υλικού αυτού του ιστοτόπου, {\bf εκτός} αν είστε καταχωρημένος κοινωφελής 
οργανισμός του οποίου ο σκοπός είναι η προώθηση της αστρονομικής επιστήμης, 
{\bf ή} έχετε τη γραπτή άδεια του συγγραφέα.

\newpage

\centerline{\includegraphics{tmp/starwheel}}

\vspace{1cm}
Ο κεντρικός τροχός αστεριών του επιπεδόσφαιρου, ο οποίος πρέπει μπει στο διπλωμένο χαρτί.

\newpage
\thispagestyle{empty}
\vspace*{-3.0cm}
\centerline{\includegraphics{tmp/holder}}
\newpage

\centerline{\includegraphics{tmp/altaz}}

\vspace{1cm}
Το πλέγμα μπορεί να εκτυπωθεί προεραιτικά σε διαφανές πλαστικό και να κολληθεί στο κομμένο κομμάτι του σώματος του επιπεδοσφαίρου για να δείχνει το υψόμετρο των αντικειμένων στον ουρανό και την κατεύθυνσή τους.

\end{document}

